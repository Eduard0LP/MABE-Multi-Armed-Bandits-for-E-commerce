% Preamble
\documentclass[../main.tex]{subfiles}

% Document
\begin{document}
\chapter*{Resumen}\label{ch:resumen}

En este documento se aborda el problema del bandido multibrazo, haciendo énfasis
en los principales algoritmos utilizados clásicamente para resolverlo y en las posibles
aplicaciones en entornos reales. Los algoritmos tratados son $\epsilon$-Greedy, 
Upper Confidence Bound, Thompson Sampling, Linear Upper Confidence Bound y Linear
Thompson Sampling, y son aplicados al caso práctico de una plataforma de
comercio electrónico, las cuales han tomado mayor relevancia con la digitalización
de nuestra sociedad. Se analiza la precisión de las recomendaciones realizadas por
los algoritmos previamente mencionados en función de si sus usuarios hacen clic o
no en los productos sugeridos por el algoritmo, se destaca el aumento de la
precisión de las recomendaciones cuando se hace uso del contexto del usuario para
realizar las sugerencias y se habla de la posibilidad de utilizar modelos basados en
aprendizaje profundo para mejorar aún más las recomendaciones del sistema, lo
cual se deja para futuros trabajos.

\textbf{Palabras clave}: bandido multibrazo, $\epsilon$-Greedy, Upper Confidence Bound, 
Thompson Sampling, Linear Upper Confidence Bound, Linear Thompson Sampling, recompensa, 
regret, comercio electrónico.

\end{document}