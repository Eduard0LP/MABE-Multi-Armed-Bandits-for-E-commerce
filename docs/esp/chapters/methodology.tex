% Preamble
\documentclass[../main.tex]{subfiles}

% Document
\begin{document}
\chapter{Metodología}\label{ch:methodology}

Como ya fue mencionado previamente, la parte práctica de este trabajo consistirá en
aplicar los bandidos multibrazo a una plataforma de comercio electrónico. En este
apartado se detallará el procedimiento seguido para llevar a cabo este caso de uso.

Como es habitual en el caso de las aplicaciones del aprendizaje por refuerzo, en vez
de usar directamente un conjunto de datos para entrenar los modelos
implementados, se construyó un simulador o bot contra el que usar los algoritmos,
es decir, este bot genera para cada paso temporal (o turno) una recompensa para la
acción escogida por el algoritmo utilizado. Una vez devuelta la recompensa, se
actualiza la política del algoritmo del bandido multibrazo sobre sus preferencias
sobre qué acción escoger en cada caso y se avanza al siguiente turno, donde el
simulador genera un nuevo usuario, el algoritmo sugiere la acción más adecuada
para ese usuario, y el simulador devuelve la recompensa de dicha acción.

En este caso, el usuario corresponderá con un usuario que ha accedido a la página
web, las acciones serán los posibles productos que recomendar, y la recompensa
será si el usuario clica o no en el producto recomendado para realizar una compra.

El simulador utilizado es de creación propia, y su funcionamiento es el siguiente:
\begin{itemize}
    \item Primero se utiliza una función para generar un conjunto de usuarios que serán
los posibles clientes. Estos usuarios tendrán asociado un id, edad, género y
una categoría de productos preferida.
    \begin{itemize}
        \item El id es un número que permitirá identificar a los usuarios de forma
unívoca.
        \item La edad se generará como un número entero aleatorio entre 15 y 70.
        \item El género será una elección aleatoria entre masculino (M) y femenino (F).
        \item Y la categoría preferida será una aleatoria de entre 5 categorías:
tecnología, moda, libros, deportes y hogar.
    \end{itemize}
    \item Se utiliza otra función diferente para generar los productos que habrá
disponibles en la plataforma de comercio electrónico. Estos productos tendrán
asociados un id, una categoría, una tasa base de clics que recibe el producto
en la plataforma y un factor de edad.
    \begin{itemize}
        \item El id en este caso, también será un número entero que permitirá identificar
los productos de forma unívoca.
        \item La categoría es la sección a la que pertenece el producto. Al igual que se
mencionó para el generador de usuarios, las categorías posibles serán:
tecnología, moda, libros, deportes y hogar. El simulador está diseñado de
forma que todas las categorías tengan el mismo número de productos.
        \item La tasa base de clics define la probabilidad base (previa a tener ningún
tipo de información del cliente o de la hora de acceso a la página web) de
que se haga clic en un determinado producto si este es recomendado a un
cliente. Estas probabilidades son generadas a partir de una distribución
uniforme con límite inferior en 0,15 y límite superior en 0,3, $U\left(0.15, 0.3\right)$.
        \item El factor de edad estima como de atractivo es un producto para los
clientes según sus edades. Este factor añade un sumando que sigue una
función lineal a la tasa base de clics (más información sobre esta más
adelante). Los valores de este factor provienen de muestras aleatorias de
una normal $N\left(0.1, 0.05\right)$ acotados inferiormente por 0,02, es decir, si
alguna de las muestras aleatorias tomadas fuera inferior a 0,02, el valor de
esa muestra sería fijado a dicho valor, y superiormente por 0,4.
    \end{itemize}
    \item El tercer y último elemento del simulador es el simulador de interacciones, el
cual se encarga de, para cada interacción usuario-producto, decidir si el
usuario clicó o no en el producto recomendado.
\end{itemize}

Esta decisión de si el usuario clicó o no en el producto recomendado se hace
tomando una muestra de una distribución Bernoulli, $Bernoulli\left(p\right)$, cuya
probabilidad se calcula como se explicará a continuación:

Para cada producto, como ya se mencionó previamente, existe una tasa base
de clics cuyos valores provienen de muestras de una distribución uniforme
continua $U\left(0.15, 0.3\right)$.

A esta tasa base se le multiplica un factor según si la categoría de productos
preferida del cliente coincide con la categoría del producto. Este factor será de
1,5 en caso de que coincidan y de 0,75 en caso de que no.

También se le multiplica un factor de género según la categoría del producto
sugerido. Si el usuario es de género masculino y el producto sugerido es de
las categorías de deportes o de tecnología se multiplicará por un factor de 1,1,
y si es de las otras 3 categorías el factor será de 0,9. En caso de que el
usuario sea de género femenino, los factores serán al revés, 0,9 para los
productos de las categorías de deportes y de tecnología, y 1,1 para las otras
tres categorías.

A esta tasa de clics también se le añaden dos sumandos adicionales, uno de
los cuales sigue una función lineal en función de la edad del usuario y el otro
sigue otra función lineal en función de la hora a la que el usuario accedió a la
página web.

Por tanto, la tasa de clics final vendrá dada por la expresión:

\begin{equation*}
    ctr = base \cdot cat \cdot gen + age + hour
\end{equation*}

Donde $base$ es la tasa base de clics, $cat \in \left\lbrace 0.75 ,1.5\right\rbrace$ es el factor de
coincidencia de categorías explicado previamente, $gen \in \left\lbrace 0.9 ,1.1\right\rbrace$ es el factor de género también previamente explicado, $age$ es el sumando lineal en
función de la edad del usuario y $hour$ es el sumando lineal en función de la
hora a la que se accedió a la página.

El sumando $age$ viene dado por la expresión:

\begin{equation*}
    age= \dfrac{a \cdot x}{70}
\end{equation*}

Donde $a \in N\left(0.1, 0.05\right)$ es el coeficiente de la regresión lineal y $x$ la edad del
usuario. El valor de $a$, además es forzado a estar acotado en el intervalo $\left[0.02, 0.4\right]$.

El sumando $hour$ viene dado por la expresión:

\begin{equation*}
    hour= \dfrac{0.05 \cdot x}{24}
\end{equation*}

donde en este caso $x$ es la hora de acceso a la página web.

Tras el diseño de este simulador, el siguiente paso fue utilizar dicho simulador como
el generador de datos para entrenar los algoritmos de aprendizaje por refuerzo
presentados en el estado del arte.

Para el caso de los algoritmos del bandido multibrazo que no utilizan el contexto ($\epsilon$-
Greedy, UCB y TS) este fue ignorado, mientras que para los algoritmos del bandido
multibrazo contextual (LinUCB y LinTS) el contexto sí que fue utilizado. Sin embargo,
este contexto, dado que contiene variables categóricas (género, categoría preferida
del usuario y categoría del producto), dichas variables tuvieron que transformarse, ya
que no pueden usarse directamente para entrenar los modelos lineales subyacentes
de estos algoritmos.

Estas variables fueron codificadas utilizando One-Hot encoding, lo cual consiste en
crear una variable binaria para cada categoría de una de estas variables
categóricas, de forma que esa variable vale 1 si el dato seleccionado pertenece a
dicha categoría y 0 en caso contrario.

Hay otras formas alternativas de codificar estas variables categóricas, como, por
ejemplo, la codificación ordinal, en la que a cada categoría de una variable se le
asigna un número diferente. Sin embargo, se decidió escoger el One-Hot encoding
ya que este no introduce ningún sesgo en las diferentes categorías de una variable,
como sí lo hace, por ejemplo, la codificación ordinal, la cual asigna un orden a las
diferentes categorías.

Antes de realizar una comparativa de los diferentes modelos, los hiperparámetros de
los algoritmos de $\epsilon$-Greedy, Upper Confidence Bound, Linear Upper Confidence
Bound y Linear Thompson Sampling fueron optimizados individualmente para poder
tener la mejor versión posible de cada uno de estos modelos de cara a hacer una
comparativa justa.

Notar que los hiperparámetros del algoritmo de Thompson Sampling no fueron
optimizados. Esto es debido a que este algoritmo carece de hiperparámetros
numéricos similares a los de los otros algoritmos. En este caso, el único
hiperparámetro es la elección de la distribución a priori utilizada, la cual se fijó a una
distribución beta.

Tras esta optimización de hiperparámetros, todos los modelos fueron entrenados a
la vez, de forma que el usuario seleccionado en cada paso temporal del problema
fuera el mismo para todos los algoritmos. Para este usuario, se simuló si haría clic
en cada uno de los diferentes productos disponibles en la plataforma si estos le
fueran sugeridos, y una vez hecho esto se llamó a los modelos para que realizaran
sus sugerencias y se devolviera el resultado del producto sugerido. Con el resultado
de la sugerencia de cada algoritmo, estos fueron actualizados (cada uno únicamente
con el resultado de su producto sugerido) y se pasó a la siguiente iteración.

Para medir el rendimiento de todos estos algoritmos tanto la recompensa acumulada
como el regret acumulado fueron utilizados. Además de estas métricas, la precisión
(número de veces que la mejor acción posible fue sugerida / número de pasos
temporales) y la proporción de accesos que terminaron en que el usuario clicara en
el producto sugerido (tasa de éxito) también fueron analizadas.
\end{document}