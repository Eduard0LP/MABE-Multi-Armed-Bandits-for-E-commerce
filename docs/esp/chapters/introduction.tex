% Preamble
\documentclass[../main.tex]{subfiles}

% Document
\begin{document}
\chapter{Introducción}\label{ch:introduction}

En este trabajo se pretende abordar el problema del bandido multibrazo y su
extensión a los bandidos multibrazo contextuales. Este problema es uno de los
ejemplos más comunes dentro del aprendizaje automático por refuerzo y cuenta con
numerosas aplicaciones en el mundo real. De estas aplicaciones, las más
extendidas son: el diseño de sistemas de recomendación (para plataformas de
``streaming'', comercio electrónico o redes sociales), ensayos clínicos, enrutamiento
dinámico y el diseño de pruebas A/B.

En el caso de los sistemas de recomendación, el auge de la cantidad de servicios
que hoy en día son realizados por la mayoría de las personas a través de internet ha
hecho que estos sistemas tomen una enorme importancia a la hora de maximizar los
beneficios para múltiples empresas y, por tanto, la aplicación de los algoritmos para
la resolución del problema del bandido multibrazo y el aprendizaje por refuerzo en
general han cobrado un mayor grado de relevancia recientemente.

La mayoría de estos servicios ofrecidos a través de internet generan una enorme
cantidad de datos sobre como los usuarios interaccionan con una página web, lo
cual provoca que este tipo de aplicaciones del problema del bandido multibrazo sean
realizadas en entornos de Big Data, conllevando ciertos retos, aunque también
ciertas ventajas, ya que los algoritmos de aprendizaje por refuerzo nunca paran de
aprender y a medida que se van generando más datos (siempre y cuando estos
datos sean de buena calidad) estos sistemas seguirán mejorando más y más cada
día.

En lo que respecta a este trabajo, se pretende realizar una introducción teórica al
problema del bandido multibrazo y su extensión al bandido multibrazo contextual, y
también a los principales algoritmos usados para resolver estos problemas,
mostrando además como se utilizarían estos algoritmos para el sistema de
recomendación de una plataforma de comercio electrónico.

El documento estará estructurado de la siguiente forma:
\begin{itemize}
\item En una primera sección se expondrán la motivación y los objetivos del trabajo.
\item A continuación, se hará una revisión de la literatura, en la que se explicará
todo el contexto teórico de estos problemas y los principales algoritmos
usados para resolverlos.
\item Después, se tendrá una sección destinada a la metodología seguida para la
implementación de la parte práctica del trabajo, en la que se aplicarán los
algoritmos presentados en el estado del arte a un sistema de recomendación
en una web de comercio electrónico.
\item Tras la metodología, habrá una sección de resultados y discusión, en la que
se explicarán los resultados obtenidos de la aplicación de estos algoritmos al
sistema de recomendación.
\item Y para finalizar, habrá un apartado de conclusiones en el que se expondrán
las conclusiones generales del trabajo.
\end{itemize}

\end{document}