% Preamble
\documentclass[../main.tex]{subfiles}

% Document
\begin{document}
\chapter{Conclusiones}\label{ch:conclusions}

En este documento se ha realizado un estudio del problema del bandido multibrazo y su 
extensión a los bandidos multibrazo contextuales, explicando con detalle los principales 
algoritmos usados tradicionalmente para sus resoluciones. 

Este tipo de problemas, como se ha podido ver en el trabajo, pueden trasladarse desde el 
plano teórico al plano práctico en múltiples ámbitos, siendo por tanto de gran utilidad.

En el caso del uso de estos algoritmos para la implementación de sistemas de recomendación,
esto supone una clara mejora de las recomendaciones con respecto a la realización de estas 
de forma completamente aleatoria, de ahí el auge de este tipo de sistemas en plataformas de
comercio electrónico, redes sociales, etc. 

En lo que respecta a los algoritmos implementados en el sistema de recomendación diseñado, 
estos o bien ignoran el contexto por completo, o lo modelan como una función lineal, lo 
cual en muchos casos prácticos no se asemeja a la realidad.

A día de hoy, el estado del arte de los sistemas de recomendación se está moviendo cada vez
más hacia el uso de modelos de aprendizaje profundo cada vez más complejos. Sin embargo, el 
uso de estos modelos clásicos siempre tendrá cabida para ciertas aplicaciones, 
especialmente aquellas en las que el contexto disponible es bastante escaso o se carece de
él, situaciones en las que el aprendizaje profundo sufre para ser capaz de dar buenos 
resultados. Por este motivo, el estudio de estas técnicas sigue siendo a día de hoy de 
vital importancia.

En lo que respecta a las técnicas de aprendizaje profundo que pueden ser usadas para 
resolver el problema del bandido multibrazo, su estudio y comparación de rendimiento con 
las técnicas estudiadas en este documento se deja para futuros trabajos.

\end{document}