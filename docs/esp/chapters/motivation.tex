% Preamble
\documentclass[../main.tex]{subfiles}

% Document
\begin{document}
\chapter{Motivación y objetivos}\label{ch:motivation}

La razón de la elección del problema del bandido multibrazo como tema de este
trabajo fin de master (TFM) es el de realizar un estudio de un campo del aprendizaje
automático (el aprendizaje por refuerzo) que normalmente no es tratado con mucho
detalle cuando se empieza a estudiar sobre el aprendizaje automático y la
inteligencia artificial, ya que el aprendizaje no supervisado, y sobre todo el
supervisado, suelen acaparar más el foco debido su mayor número de aplicaciones
en diferentes ámbitos.

El principal objetivo de este trabajo consiste en mostrar la utilidad del marco teórico
de los bandidos multibrazo para la resolución de problemas del mundo real, como el
diseño de un sistema de recomendación para una plataforma de comercio
electrónico parecido a lo realizado en \textcite{rohde2018}, que se verá en la parte práctica
de este trabajo.

Los problemas del bandido multibrazo y el bandido multibrazo contextual son
problemas que suelen ser estudiados desde un punto de vista académico como un
primer ejemplo de aprendizaje por refuerzo, pero no suele llegar a estudiarse nunca
su traslación a entornos del mundo real. Estos problemas, a pesar de su aparente
simplicidad, dan lugar a un campo de estudio amplio y complejo, y cuya aplicación
en entornos empresariales puede tener un alto valor añadido.
\end{document}